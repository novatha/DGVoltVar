% - Giving a talk on some subject.
% - The talk is between 15min and 45min long.
% - Style is ornate.

% MODIFIED by Jonathan Kew, 2008-07-06
% The header comments and encoding in this file were modified for inclusion with TeXworks.
% The content is otherwise unchanged from the original distributed with the beamer package.

\documentclass[10pt]{beamer}


% Copyright 2004 by Till Tantau <tantau@users.sourceforge.net>.
%
% In principle, this file can be redistributed and/or modified under
% the terms of the GNU Public License, version 2.
%
% However, this file is supposed to be a template to be modified
% for your own needs. For this reason, if you use this file as a
% template and not specifically distribute it as part of a another
% package/program, I grant the extra permission to freely copy and
% modify this file as you see fit and even to delete this copyright
% notice. 


\mode<presentation>
{
  \usetheme{Montpellier}
  % or ...

  \setbeamercovered{transparent}
  % or whatever (possibly just delete it)
}


\usepackage[english]{babel}
% or whatever

\usepackage{bibentry}
% for bib on footnote  http://www.latex-community.org/forum/viewtopic.php?f=4&t=12027

%\usepackage[utf8]{inputenc}
% or whatever

\usepackage{times}
\usepackage[T1]{fontenc}
% Or whatever. Note that the encoding and the font should match. If T1
% does not look nice, try deleting the line with the fontenc.

\usepackage{booktabs}
% for beautiful table
\title[ DG participation in DS control] % (optional, use only with long paper titles)
%{Distribution System  Volt/Var Control in the Presence of DGs}
{DG Operation for Distribution System  Volt/Var Control}
\subtitle
{} % (optional)

\author[N. Daratha] % (optional, use only with lots of authors)
{N.~Daratha }
% - Use the \inst{?} command only if the authors have different
%   affiliation.

\institute[IIT Roorkee] % (optional, but mostly needed)
{

guided by \\ \vspace{.10cm}Prof. J.D. Sharma and Prof. B. Das\\
\vspace{.10cm}  Department of Electrical Engineering\\
IIT Roorkee
}
  
% - Use the \inst command only if there are several affiliations.
% - Keep it simple, no one is interested in your street address.

\date[Short Occasion] % (optional)
{PhD Seminar Course}

\subject{Talks}
% This is only inserted into the PDF information catalog. Can be left
% out. 



% If you have a file called "university-logo-filename.xxx", where xxx
% is a graphic format that can be processed by latex or pdflatex,
% resp., then you can add a logo as follows:

% \pgfdeclareimage[height=0.5cm]{university-logo}{university-logo-filename}
% \logo{\pgfuseimage{university-logo}}



% Delete this, if you do not want the table of contents to pop up at
% the beginning of each subsection:
\AtBeginSubsection[]
{
  \begin{frame}<beamer>{Outline}
    \tableofcontents[currentsection,currentsubsection]
  \end{frame}
}


% If you wish to uncover everything in a step-wise fashion, uncomment
% the following command: 

%\beamerdefaultoverlayspecification{<+->}

\setbeamertemplate{footline}[page number]
\begin{document}
\def\newblock{\hskip .11em plus .33em minus .07em} % for bibliography and newblock
\bibliographystyle{ieeepes}
\nobibliography{./dpf}

\begin{frame}
  \titlepage
\end{frame}
%%
\begin{frame}{Proposition}
\begin{center}
There is a need for an effective methodology of multi-objective  \textbf{variable-power-factor distributed generation operation} for distribution system volt/var control during normal and emergency situation.\\ 
\end{center}
\end{frame}



\begin{frame}{Outline}
  \tableofcontents
  % You might wish to add the option [pausesections]
\end{frame}


% Since this a solution template for a generic talk, very little can
% be said about how it should be structured. However, the talk length
% of between 15min and 45min and the theme suggest that you stick to
% the following rules:  

% - Exactly two or three sections (other than the summary).
% - At *most* three subsections per section.
% - Talk about 30s to 2min per frame. So there should be between about
%   15 and 30 frames, all told.
%\part{Introduction}
\section{Electric Distribution System}

%\begin{frame}[Distribution System]{Electric Power System}{Distribution System}
%\begin{figure}
%	\centering
%		\includegraphics[height=4.5cm]{images/ElectricPowerSystem2.pdf} 
%	\label{fig:DFIG}
%\end{figure}
%\end{frame}


%\begin{frame}[Elements of Distribution Systems]{Elements of Distribution Systems}
%Excluding DG, ....
%\begin{itemize}
%\item
%All DS must have feeders and transformer with On-load Tap Changer.
%\item Most of them have shunt capacitors and/or shunt reactor
%\item fewer of them have SVC (static Var compensator)
%\item even fewer of them have D-STATCOM.
%\end{itemize}
%\end{frame}


\begin{frame}{Variation! Variation! Variation!}[Distribution System]
\begin{figure}
	\centering
		\includegraphics[height=6.5cm]{images/diagramumum.pdf} 
	\label{fig:umum}
\end{figure}
\end{frame}


\begin{frame}{}[Distribution System]
\begin{figure}
	\centering
		\includegraphics[height=6.5cm]{images/awarenesshot.pdf} 
	\label{fig:aware}
\end{figure}
\end{frame}


%\subsection{Multi-Objectiveness}
%\begin{frame}{We Want Many Objectives}
%A distribution system must
%\begin{itemize}
%\item
%have good voltage regulation
%\item be energy efficient
%\item
%have wide stability margin
%\item support transmission system reactive power need
%\item of course, maximize overall profit.
%\end{itemize}
%However, achieving all of them at the same time is NOT possible.
%\end{frame}

%\begin{frame}[Nature of Line]{Feeders: Minimum Losses $\neq$  Minimum Voltage Drop}
%\begin{itemize}
%\item Feeders bring electricity to consumers.
%\item
%A feeder power loss is minimum when when load is pure resistive.
%\item A feeder voltage drop is minimum when load capacitive reactive power equals feeders requirement.
%\end{itemize}
%\end{frame}

\begin{frame}{Control Devices in Distribution Systems}
For effective, secure, and safe operation of DS, utility control:
\begin{itemize}
\item Switches
\item Voltage regulators (OLTC, SC, SR)
\item Distributed Generators
\item Energy Storages
\end{itemize}
\end{frame}


\section{Distributed Generation}

%\subsection[Definition and Classification]{Definition and Classification}

\begin{frame}{Distributed Generation (DG)}{definition, altenative names}
  % - A title should summarize the slide in an understandable fashion
  %   for anyone how does not follow everything on the slide itself.
  \begin{itemize}
  \item
    a distributed generation (DG) is a small generation connected to distribution network
 \item IEEE Standard Dictionary Terms : %\footnote{\tiny \bibentry{4116787}}:
  \begin{quote}
    Electric generation facilities connected to an Area EPS (Electric Power System) through a PCC (Point of Common Copling);
    a subset of DR (Distributed Resources).
   \end{quote}
   \item 
    alternative names: embedded generation, dispersed generation
   \end{itemize}
\end{frame}

\begin{frame}{Distributed Generation (DG)}{International Energy Agency's Definition \footnote{\tiny \bibentry{IEA2002}}}
\begin{itemize}
\item
\textbf{Distributed generation} is generating plant serving a customer on-site
or providing support to a distribution network, connected to the grid at
distribution-level voltages.
\item
\textbf{Dispersed generation} is distributed generation plus wind power and
other generation, either connected to a distribution network or completely
independent of the grid.
\end{itemize}
\end{frame}

\begin{frame}{DG Classifications}

  DGs can be\dots
  \begin{itemize}
  \item  renewable (wind,PV,hydro) or  non renewable (diesel)
  \item   dispatchable (diesel, micro/small hydro) or not-dispatchable (wind, PV)
  \item    intermittent (PV, wind, ocean wave) or        steady (diesel, hydro, fuel cell)
  \item   grid-connected or isolated
\end{itemize}
\end{frame}


%\subsection[DG grid  Connection]{DG grid  Connection}

\begin{frame} {DG-to-Power Grid Interface}

\begin{center}
\begin{tabular}{lll}
\toprule
DG Type & Electric Machine & Interface \\
\midrule
ICE & SG & directly \\
& IG & directly \\
Gas Turbines & SG & directly \\
Micro-turbines & PMSG & rectifier+inverter or\\ &&
 AC/AC converter \\
 & Squirrel cage IG & directly \\
Wind & DFIG & rectifier+ inverter \\
 & SG or PMSG & rectifier + inverter\\
Photovoltaic && inverter\\
Fuel cell&  & inverter\\
\bottomrule
\end{tabular}
\end{center}
\footnote{ICE=Internal Combustion Engine; SG=Synchronous Generator; IG= Induction Generator; PMSG = Permanent Magnet SG; DFIG=Doubly Fed IG}
\end{frame}

%\begin{frame}{DG Impacts on Voltage Regulation}{Before Fault}
%\begin{figure}
%	\centering
%		\includegraphics[height=4.5cm]{images/VR1.pdf} 
%	\label{}
%\end{figure}
%
%\end{frame}
%
%\begin{frame}{DG Impacts on Voltage Regulation}{After Fault}
%
%\begin{figure}
%	\centering
%		\includegraphics[height=4.5cm]{images/VR2.pdf} 
%	\label{}
%\end{figure}
%
%\end{frame}
%
%\begin{frame}	{DG May Not Participate in Voltage Regulation}
%IEEE Standard 1547-2003:\\ \vspace{.2cm}
%\begin{quotation}
%4.1.1 Voltage regulation\\
%The DR shall \textbf{not actively regulate} the voltage at the PCC. The DR shall \textbf{not cause} the Area EPS service
%voltage at other Local EPSs to go outside the requirements of ANSI C84.1-1995, Range A.
%\end{quotation}
%\end{frame}



%\begin{frame}{a DG Must Detect and Disconnect from Islanded DS}
%IEEE Standard 1547-2003:\\ \vspace{.2cm}
%\begin{quote}
%4.4.1 Unintentional islanding\\ \vspace{.5cm}
%For an unintentional island in which the DR energizes a portion of the Area EPS through the PCC, the DR
%interconnection system shall \textbf{detect} the island and \textbf{cease to energize} the Area EPS within two seconds of the
%formation of an island.\\\vspace{.5cm}
%4.4.2 Intentional islanding\\ \vspace{.5cm}
%This topic is \textbf{under consideration} for future revisions of this standard.
%\end{quote}
%\end{frame}




%\subsection[DG's Reactive Power Capability]{DG's Reactive Power Capability}
\begin{frame}{Some DGs Reactive Power Capability }
\begin{itemize}
\item Interface that can control reactive power :

	\begin{itemize}
		\item  synchronous machine \footnote{\bibentry{4181375}} (hydro,diesel) 
		\item  voltage source converter (PV, DFIG\footnote{\bibentry{5611583}}, Ocean Energy)
	\end{itemize}

%\item  Interface that cannot control reactive power: asynchronous machine.
\end{itemize}

\begin{figure}
	\centering
		\includegraphics[height=3cm]{images/DFIG.pdf}
	\label{fig:DFIG}
\end{figure}

\end{frame}

%\begin{frame}{DGs Have Low Utilization Level}
%\begin{itemize}
%\item PV depends on solar irradiance.
%\item Wind generator depends on wind speed.
%\item Both solar irradiation and wind speed is highly intermittent
%\item There is significant fraction of the time when DG works much below rated power.
%\item During those time, DGs can provide reactive power service.
%\end{itemize}
%\end{frame}

\begin{frame}{Distributed Reactive Power Generation Control for Voltage Rise Minimization in Distribution Network\footnote{\bibentry{4472349}}}
\begin{itemize}
\item Prevent significant voltage rise because of DG presence.
\[
Q_G^*  \approx \frac{X}{R^2 + X^2} - \sqrt{\frac{X}{R^2 + X^2}^2 - P_G^2 + \frac{2RP_G}{R^2 + X^2}  }
\]
\item Compared with constant power factor approach.
\item Effective reactive power control with two consequences:
   \begin{itemize}
   \item increased stress on tap changers.
   \item increased feeder losses.
   \end{itemize}
\item Voltage become almost independent of DG real power generation.
\item Voltage dependence on load is almost unchanged.
\end{itemize}
\end{frame}


%\begin{frame}{Voltage Become Almost Independent of DG Real Power Generation}
%\begin{figure}
%	\centering
%		\includegraphics[height=5cm]{images/voltagerise1.pdf}
%	\label{fig:DFIG}
%\end{figure}
%\end{frame}
%
%
%\begin{frame}{Voltage Dependence on Load is Almost Unchanged}
%\begin{figure}
%	\centering
%		\includegraphics[height=5cm]{images/voltagerise2.pdf}
%	\label{fig:DFIG}
%\end{figure}
%\end{frame}


\begin{frame} {Grid Interconnection of Renewable Energy Sources at Distribution Level with Power Improvement Features \footnote{\bibentry{5617328}}}
\begin{itemize}
\item Some other functions that can be provided by DGs:
\begin{itemize}
\item power transfer at unity power factor
\item local reactive power support
\item harmonic mitigation
\item load balancing
\end{itemize}
\item Those functions can be achieved simultaneously or individually
\item no additional hardware is needed
\end{itemize}
\end{frame}



\begin{frame}{Observation I}
\begin{itemize}
\item DG can cause voltage rise on the feeder to which it is connected.
\item There is a method to mitigate the voltage rise 
   \begin{itemize}
    \item variable power factor operation
    \item increased number of switching and losses.
    \end{itemize}
\item Current grid code do not allowed DG to control its output voltage.
\item DGs is also potential to improve power quality.
\end{itemize}
\end{frame}



\section{Volt/Var Control In Distribution System With DGs}

\begin{frame}
\begin{center}
\huge
Works in which DGs are in \textbf{constant power factor mode}.
\end{center}
\end{frame}

\begin{frame} {Optimal Distribution Voltage Control and coordination with distributed generation \footnote{\bibentry{4374135}}}
\begin{itemize}
\item Minimize total losses and voltage deviation
\item Control OLTC, Shunt Capacitor (SC), Shun Reactor (SR), Step Voltage Regulator (SVR), Static Voltage Controller (SVC)
\item Optimization methods : Genetic Agorithm
\item DGs = PVs with constant unity power factor.

\item Centralized control
\end{itemize}
\end{frame}

\begin{frame} {Optimal Distribution Voltage Control and coordination with distributed generation}
\begin{itemize}
\item Objective: $\min \sum w_1 |V_{n,ref} -V_n|+w_2Loss$
\item Contraints:
    \begin{itemize}
       \item voltage limits
       \item tap position limits (OLTC)
    \end{itemize}
\item Optimization methods : Genetic Agorithm
\end{itemize}
\end{frame}

%\begin{frame}{Optimal Distribution Voltage Control and coordination with distributed generation}{SVC Model}
%\begin{figure}
%	\centering
%		\includegraphics[height=3.5cm]{images/SVC.pdf}
%	\label{fig:DFIG}
%\end{figure}
%\end{frame}
%
%
%\begin{frame}{Optimal Distribution Voltage Control and coordination with distributed generation}{SVR Model}
%\begin{figure}
%	\centering
%		\includegraphics[height=3.5cm]{images/LRT.pdf}
%	\label{fig:LRT}
%\end{figure}
%\end{frame}

\begin{frame} {Works in Which DGs are in CONSTANT power factor mode 1}
\begin{itemize}
\item \bibentry{Casavola201125} $\rightarrow$ OLTC only
\item \bibentry{915503} $\rightarrow$ OLTC only
\end{itemize}
\end{frame}

\begin{frame}{Works in Which DGs are in CONSTANT power factor mode 2}
\begin{itemize}
\item \bibentry{Viawan2007} $\rightarrow$ OLTC and SC
\item %Design of The Optimal ULTC parameters in Distribution System with Distributed Generations 
\bibentry{4717276} $\rightarrow$ OLTC only
\item all of them do not include SVC and D-STATCOM
\end{itemize}
\end{frame}

\begin{frame}{Observation II: Constant Power Factor Operation}
Among paper considering DG \textbf{constant} power factor operation:
\begin{itemize}
\item most include OLTC and DG
\item other  also include SC
\item only one include SVR and SVC
\item none include D-STATCOM
\item single objective mathematical programming
\end{itemize}
\end{frame}
%\subsection[Grid Code for Voltage Regulation]{Grid Code for Voltage Regulation}
%\subsection[Voltage Regulator]{Voltage Regulators}
%\subsection[DG Participation in Voltage Regulation]{DG participation in Voltage Regulation}

\section{DG Participation in Volt/Var Control}

\begin{frame}
\begin{center}
\huge
Works in which DGs are in \textbf{variable power factor mode}
\end{center}
\end{frame}



\begin{frame}{Minimizing Reactive Power Support for Distributed Generation\footnote{\bibentry{5738712}}}
\begin{itemize}
\item Choosing power factor of DGs and setting of OLTC
\item Maximising DG reactive power generation
\item Reducing transmission system burden
\item Enhanced passive approach vs active approach
\item Uses DG and OLTC only
\end{itemize}
\end{frame}

\begin{frame}{Multiagent Dispatching Scheme for DGs for Voltage Support on Distribution Feeders\footnote{\bibentry{4077092}}}
\begin{itemize}
\item Each generator control its output based on local measurements.
\item Those measurements used to calculate sensitivity factors.
\item Coordination between DGs through a Control Net Protocol (CNP)
\item Reliable communication network
\item Uses DG and OLTC only
\end{itemize}
\end{frame}

%\begin{frame}{Options for Controls of Reactive Power by Distributed PV Generators \footnote{\bibentry{5768094}}}
%\begin{itemize}
%\item Local control of PV generators
%\item Local measurements were sufficient for voltage regulation
%\item Support the idea of Baran and Markabi (2007)
%\item Uses DG and OLTC only
%\end{itemize}
%\end{frame}

\begin{frame}{Voltage and Reactive Power Control in Systems with Synchronous Machine-Based Distributed Generation\footnote{\bibentry{4443852}}}
\begin{itemize}
\item Minimize total losses.
\item Include OLTC and SC.
\item DG regulate voltage at point of common connection.
\item If SC is enough, DG participation does not reduce losses significantly.
\item Excess reactive power can support transmission system (Ochoa, et. al. , 2011).
\end{itemize}
\end{frame}




\begin{frame}{Short-Term Schedulling and Control of Active Distribution Systems with High Penetration of Renewable Energy Resources\footnote{\bibentry{5546901}} }
\begin{itemize}
\item a day-ahead scheduler + intra-day (15 minutes) scheduler.
\item includes dispatchable and not-dispatchable DGs.
\item a day-ahead scheduler is a forecaster of generator and energy storage.
\item intraday scheduler minimize generation deviation define by the other scheduler.
\end{itemize}
\end{frame}


%\begin{frame}
%\begin{figure}
%	\centering
%		\includegraphics[height=7cm]{images/2stageERS.pdf}
%	\label{fig:2stageERS}
%\end{figure}
%\end{frame}
%
%\begin{frame} {The Day-Ahead Scheduler}
%\begin{itemize}
%\item objective is minimal energy cost  \[ \min \sum_{r=1}^R \sum_{j=1}^N c_{j,r} \Delta t P_j^r\]
%\item constraints:
%  \begin{itemize}
%  \item Electrical Load balance
%  \item Storage units
%  \item Power and energy limits
%  \item Thermal load balance
%  \end{itemize}
%\item inputs: load forecast, generation forecast, energy cost, limits of generating units, initial status of storage units.
%\end{itemize}
%\end{frame}
%
%\begin{frame}{The Intra-day Scheduler}
%\begin{itemize}
%\item Multiobjective:
%\[ \min_{\Delta x} \left[	  \sum \alpha S_P  +\beta P_{loss} + \sum \gamma  S_V \right] \]
%   \begin{itemize}
%      \item minimal voltage deviation
%      \item minimal generation deviation
%      \item minimal network losses
%   \end{itemize}
%\item Input: 15-minutes ahead forecast, state estimation results
%\item output: control signal for OLTC, voltage regulators, DGs and energy storages
%\item controlled variable: active and reactive power generation and OLTC tap position
%\end{itemize}
%\end{frame}

%\begin{frame}
%\begin{figure}
%	\centering
%		\includegraphics[height=5cm]{images/intradaysche.pdf}
%	\label{fig:intradaysche}
%\end{figure}
%\end{frame}

\begin{frame}{What are missing?}
Further considerations are needed:
\begin{itemize}
\item switching seguence?
\item transition cost?
\item security?
\end{itemize}
\begin{figure}
	\centering
		\includegraphics[height=7cm]{images/optimumpath.pdf}
	\label{fig:optimumpath}
\end{figure}
\end{frame}

\begin{frame}
\begin{center}
\huge Reducing Number of Switching: \textbf{1. Constraint Addition}
\end{center}
\end{frame}

\begin{frame}{Importance of Switching Reduction}
\begin{itemize}
\item switching may initiate transients
\item device has limited  total number of switchings
\item DG's variable power factor mode increase OLTC's switching numbers
\item slow mechanical switch vs fast load change and intermitent renewables
\end{itemize}
\end{frame}


\begin{frame}{Reactive Power and Voltage Control in Distribution System with Limited Switching Operation \footnote{\bibentry{4808225}}}
\begin{itemize}
\item Objective : min energy losses \[  \min E = \sum_{t=0}^{23} f(x_1(t),x_2(t),x_3(t)) \]
\item $x_1$ discrete variables: OLTCs and Capacitors
\item $x_2$ Q and V
\item $x_3$ P and $\theta$
\end{itemize}
\end{frame}

\begin{frame}{Reactive Power and Voltage Control in Distribution System with Limited Switching Operation}
Constraints:
\begin{itemize}
\item power flow equations
\item tap positions limits
\item capacity limits
\item additional constraints : Maximum Allowable daily switching operation (MADSON)
\[ h(x_1(0),x_1(1),..., x_1(23)) = \sum_{t=0}^{23} |x_{1(t+1)} -x_{1(t)}| \leq S_{x1}C_{x1}\]
\end{itemize}
\end{frame}

%\begin{frame}{Reactive Power and Voltage Control in Distribution System with Limited Switching Operation}
%Proposed optimization method:
%\begin{itemize}
%\item discrete variables are treated as continous variables
%\item inequality constraints are converted into equality constraints with help from slack variables
%\[x_{1(t)}+s_{u1(t)}=x_{1(t)max}\]
%\[x_{1(t)} - s_{u1(t)}=x_{1(t)min}\]
%\[x_{2(t)}+s_{u2(t)}=x_{2(t)max}\]
%\[x_{2(t)} - s_{u2(t)}=x_{2(t)min}\]
%\[h(x_{1(0)},x_{1(2)},...,x_{1(23)})=S_{x_1}C_{x_2}\]
%\[s_{u1(t)}, s_{l1(t)}, s_{u2(t)}, s_{l1(t)} \geq 0 \]
%\end{itemize}
%\end{frame}
%
%\begin{frame}{Reactive Power and Voltage Control in Distribution System with Limited Switching Operation}
%Proposed optimization method:
%\begin{itemize}
%\item interior point method was used
%\item KKT are derived and solved with Newton-Raphson method.
%\item compared with Genetic Algorithm, BARON and DICOPT
%\item test cases: Baran and Wu 69-buses system and chinese 14-buses system
%\item the proposed method is faster than other methods.
%\end{itemize}
%\end{frame}
%
%
%\begin{frame}
%\begin{figure}
%	\centering
%		\includegraphics[height=5cm]{images/tableMADSON.pdf}
%	\label{fig:tableMADSON}
%\end{figure}
%\end{frame}


\begin{frame}
\begin{center}
\huge Alternative Approach: \textbf{Rule-based Control}
\end{center}
\end{frame}

\begin{frame}{Reasons for Alternative Approach}
\begin{itemize}
\item Our problem is NP-hard MINLP unless some simplification is assumed.
\item Distribution system is large
\item Slow voltage controller movement and changing load and generation profile
\item minimum switching is favorable
\item some switching action are mutually exclusive
\end{itemize}
\end{frame}

\begin{frame}{Configurable, Hierarchical, Model-Based Control of Electrical Distribution Circuits\footnote{\bibentry{5604720}}  }
\begin{itemize}
 \item objective : close and better operating state; minimize change of state
 \item preference-based multi objectives and constraints:
  \begin{itemize}
    \item voltage regulation
    \item Capacity constraint
    \item losses
    \item priority is adjustable
  \end{itemize}
 \item control devices : SC, OLTC, SVR, DG
   \begin{itemize}
    \item single step (SS) : SC, DG (on-min-on)
    \item multi step (MS) : OLTC, DG (min - max discretized)
    \end{itemize}

\end{itemize}
\end{frame}
%
%\begin{frame}{CHMC Main Loop}
%\begin{figure}
%	\centering
%		\includegraphics[height=6cm]{images/CHMC1.pdf}
%	\label{fig:CHMC1}
%\end{figure}
%\end{frame}
%
%\begin{frame}{CHMC Main Loop}
%\begin{figure}
%	\centering
%		\includegraphics[height=6cm]{images/CHMC2.pdf}
%	\label{fig:CHMC2}
%\end{figure}
%\end{frame}
%
%\begin{frame}{Selection of New State}
%\begin{figure}
%	\centering
%		\includegraphics[height=6cm]{images/CHMCdecision1.pdf}
%	\label{fig:CHMCdecision1}
%\end{figure}
%If voltage deviation is smaller than before, accept this newer state. 
%\end{frame}

%\begin{frame} {Selection of New State}
%\begin{figure}
%	\centering
%		\includegraphics[height=6cm]{images/CHMCdecision2.pdf}
%	\label{fig:CHMCdecision2}
%\end{figure}
%\end{frame}

%\begin{frame} {Selection of New State}
%\begin{figure}
%	\centering
%		\includegraphics[height=6cm]{images/CHMCdecision3.pdf}
%	\label{fig:CHMCdecision3}
%\end{figure}
%\end{frame}



\begin{frame}{Ways to Reduce Number of Switching}
Using previous methods, variable power factor DGs operation increase number of switching. There are to ways to reduce the number:
\begin{itemize}
\item MADSON constraint 
\item rule-based optimization
\end{itemize}
\end{frame}


\begin{frame}{Observation III: Variable Power Factor Operation}
Among paper considering DG \textbf{variable} power factor operation:
\begin{itemize}
\item most include only OLTC and DG (one include DG)
\item single-objective mathematical programming
\item increased number of switching is expected
\end{itemize}
\end{frame}

%
%\begin{frame}{Observation IV: Possible Gaps for Future Research}
%What is not available in literature is volt/var control strategy/method which:
%\begin{itemize}
%\item include a rather complete types of (potential) voltage regulator
%\item is multi-objective optimization plus higher information processing
%\end{itemize}
%In addition, optimum switching sequence needed to reach the optimum state has not been well studied.
%\end{frame}


\begin{frame}{}
\begin{center}
\huge{Research Gap}
\end{center}
\end{frame}
\begin{frame}{Research Gaps: SVC}
\begin{enumerate}
\item They are available.
\item SVCs are much faster than OLTC and SC.
\item DG intermittence may be to fast for OLTC.
\item Coordination is not extensively studied.
\item Previous coordination method assume single phase modelling   \footnote{\bibentry{4374135}}
\end{enumerate}
\end{frame}


\begin{frame}{Why Three Phase Modeling?}
\begin{enumerate}
\item Distribution Feeders are not transposed.
\item There are many single phase load and feeder.
\item Small-scale DG is single phase.
\item Unbalance can cause voltage violations.
\item non-uniform OLTC parameters setting is better that uniform  \footnote{\bibentry{4717276}}.
\end{enumerate}
\end{frame}

\begin{frame}{Why DG?}
\begin{enumerate}
\item DG penetration is increasing
\item DG reverse power flow is creating new situation
\item Renewable Energy is intermittent
\item Simulated  DG + OLTC coordination proved to be beneficial.
\end{enumerate}
\end{frame}

%\section{Mathematical Programming Consideration}
%%\subsection{Nature of The Problem}

%%\subsection{Model Completeness}


\section*{Summary}

\begin{frame}{Summary}
  % Keep the summary *very short*.
  \begin{itemize}
  \item DGs reactive power capability is not fully utilised.
  \item Grid codes require  constant-power factor operation.
  \item Most published research follow the grid codes.
  \item Some works consider the variable-power  factor (VPF) operation.
  \item VPF operation increase number of switchings of voltage regulators
  \item two ways in limiting switching number: MADSON constraint and a rule-based approach

  \end{itemize}  
%  % The following outlook is optional.
%%  \vskip0pt plus.5fill
%  \begin{itemize}
%  \item
%    Outlook\\
%\textit{multiobjective volt/var control using modern and conventional regulators and DGs may not be well studied.}
%  \end{itemize}
\end{frame}

\begin{frame}
\begin{center}
\textbf{Thank You Very Much}
\maketitle
\end{center}

 % \titlepage
\end{frame}



\end{document}


